% #####################################################################
% #####################################################################
% ##                                                                 ##
% ##                             Lizenz:                             ##
% ##                         CC BY-NC-SA 3.0                         ##
% ##      http://creativecommons.org/licenses/by-nc-sa/3.0/de/       ##
% ##                                                                 ##
% #####################################################################
% ##   Diese Datei kann beliebig verändert werden, solange darauf    ##
% ##     hingewiesen wird, dass dieses Dokument ursprünglich von     ##
% ##                                                                 ##
% ##                        www.ei-studium.de                        ##
% ##                                                                 ##
% ##                             stammt.                             ##
% ## Dies gilt insbesondere auch für alle daraus erstellten Dateien. ##
% ##    Des Weiteren muss die Weitergabe dieser Dateien unter der    ##
% ##                    gleichen Lizenz erfolgen.                    ##
% #####################################################################
% #####################################################################
\documentclass[a4paper,twocolumn,10pt]{article}
\usepackage[utf8]{inputenc}
\usepackage[ngerman]{babel}
\usepackage[top=2.0cm,bottom=1.5cm,left=1.0cm,right=1.0cm]{geometry}
\usepackage{enumitem}
\usepackage{graphicx}
\usepackage{amsfonts}
\usepackage{amsmath}
\usepackage{sectsty}
\usepackage{colortbl}
\usepackage{cancel}
\usepackage{listings}
\usepackage{color}
\usepackage{amsmath}
\usepackage{fancyhdr}
\usepackage[pdfborder={0 0 0}]{hyperref}

\setlist{itemsep=.01mm}
\setenumerate{label=\emph{\arabic*})}
\setlength{\columnsep}{1cm}
\parindent 0mm

\partfont{\huge}
\sectionfont{\Large \sc\bf}
\subsectionfont{\normalsize}
\subsubsectionfont{\small\textit}

\pagestyle{fancy}
\lhead[\leftmark]{Elektrizität und Magnetismus - Formeln}
\chead[\leftmark]{\url{http://www.ei-studium.de}}
\rhead[\leftmark]{Erstelldatum: \today}
\lfoot[\leftmark]{Keine Garantie auf Vollständigkeit und Richtigkeit!}
\cfoot[\leftmark]{}
\rfoot[\leftmark]{\thepage}
\renewcommand{\headrulewidth}{0.5pt}
\renewcommand{\footrulewidth}{0.5pt}

\begin{document}

\section{Elektrizität und Magnetismus}

\subsection{Mathematik}
\begin{enumerate}[label=$\bullet$]
\item Wegintegral:
\begin{equation*}
 \int\limits_{C(P_1,P,2)}\overrightarrow{F}(\overrightarrow{r})d\overrightarrow{r}=\int\limits_{\lambda_0}^{\lambda_1}\overrightarrow{F}(\overrightarrow{r}(\lambda))\frac{d\overrightarrow{r}}{d \lambda}d \lambda
\end{equation*}
\item Integrabilitätsbedingung:
\begin{equation*}
\frac{\partial E_j}{\partial x_i}=\frac{\partial E_i}{\partial x_j}\Leftrightarrow \overrightarrow{F}(\overrightarrow{r})\;konservativ
\end{equation*}
\item Zylinderkoordinaten:
\begin{equation*}
h_r=1;\;\;h_{\varphi}=r;\;\;h_z=1;\;\;\varphi\in [0,2\pi]
\end{equation*}
\item Kugelkoordinaten:
\begin{equation*}
h_r=1;\;\;h_{\theta}=r;\;\;h_{\varphi}=r\sin(\theta)
\end{equation*}
\item Gradient:
\begin{equation*}
\nabla\Phi =\sum\limits_{i=1}^{n}\frac{1}{h_i}\frac{\partial\Phi}{\partial u_i}\overrightarrow{e}_{u_i}
\end{equation*}
\item Integralsatz von Gauß:
\begin{equation*}
\int\limits_{V}\nabla\overrightarrow{U}dv=\int\limits_{\partial V}\overrightarrow{U}d\overrightarrow{a}
\end{equation*}
\item Integralsatz von Stokes:
\begin{equation*}
\int\limits_{A}rot(\overrightarrow{U})d\overrightarrow{a}=\int\limits_{\partial A}\overrightarrow{U}d\overrightarrow{r}
\end{equation*}
\end{enumerate}

\subsection{Elektrostatik}
\begin{enumerate}[label=$\bullet$]
\item
\begin{equation*}
\overrightarrow{F_q}(\overrightarrow{r})=\frac{q}{4\pi\epsilon}\sum\limits_{i=1}^{N}\frac{q_i}{|\overrightarrow{r}-\overrightarrow{r}_i|^3}\cdot (\overrightarrow{r}-\overrightarrow{r}_i)
\end{equation*}
\item
\begin{equation*}
\overrightarrow{F}_q(\overrightarrow{r})=q\cdot \overrightarrow{E}(\overrightarrow{r})
\end{equation*}
\item
\begin{equation*}
\overrightarrow{E}(\overrightarrow{r})=\frac{1}{4\pi\epsilon}\cdot\int\limits_{E_3}\frac{\overrightarrow{r}-\overrightarrow{r'}}{|\overrightarrow{r}-\overrightarrow{r'}|^3}\rho(\overrightarrow{r'})d^3r'
\end{equation*}
\item
\begin{equation*}
rot(\overrightarrow{E})=0\Leftrightarrow\overrightarrow{E}\;konservativ\Leftrightarrow\oint \overrightarrow{E}d\overrightarrow{r}=0
\end{equation*}
\item Coulomb-Potential:
\begin{equation*}
\Phi(\overrightarrow{r})=\frac{1}{4\pi\epsilon}\cdot\sum\limits_{i=1}^{N}\frac{q_i}{|\overrightarrow{r}-\overrightarrow{r_i}|}=\frac{1}{4\pi\epsilon}\int\limits_{E_3}\frac{\rho(\overrightarrow{r'})}{|\overrightarrow{r}-\overrightarrow{r'}|}d^3r'
\end{equation*}
\item 
\begin{equation*}
\overrightarrow{E}=-\nabla\Phi
\end{equation*}
\item
\begin{equation*}
\nabla\overrightarrow{D}=\rho\Rightarrow\nabla(\epsilon\overrightarrow{E})=\rho\Rightarrow\nabla\nabla\Phi=-\frac{\rho}{\epsilon}
\end{equation*}
$\Rightarrow$ Poissongleichung: $\Delta\Phi=-\frac{\rho}{\epsilon}$\\\\
(Lösung: Coulomb-Potential im $E_3$)
\item
\begin{equation*}
U_{12}=\Phi(P_1)-\Phi(P_2)=\int\limits_{P_1}^{P_2}\overrightarrow{E}d\overrightarrow{r}
\end{equation*}
\item
\begin{equation*}
\Phi(\overrightarrow{r})=\Phi(\overrightarrow{r}_0)-\int\limits_{P_0}^{P_1}\overrightarrow{E}d \overrightarrow{r}
\end{equation*}
\item
\begin{equation*}
W_{12}=\int\limits_{C}\overrightarrow{F}d\overrightarrow{r}=q\cdot U_{12}
\end{equation*}
\item
\begin{equation*}
\overrightarrow{D}\cdot\overrightarrow{N}=\sigma
\end{equation*}
\item
\begin{equation*}
\sigma =\frac{Q}{A}
\end{equation*}
\item
\begin{equation*}
 Q=\int\limits_{A}\sigma(\overrightarrow{r})da=\sum\limits_{\overrightarrow{r_i}\in V}q_i
\end{equation*}
\item Kapazität:
\begin{equation*}
C=\frac{Q}{U}
\end{equation*}
\item Kugelkondensator:
\begin{equation*}
C=4\pi\epsilon\frac{a\cdot b}{b-a}
\end{equation*}
\item
\begin{equation*}
W_{el}=\int\limits_{0}^{Q}u(q)dq=\frac{1}{2}CU^2=\frac{1}{2}\frac{Q^2}{C}
\end{equation*}
\item Elektrische Energiedichte:
\begin{equation*}
w_{el}=\frac{1}{2}\cdot\overrightarrow{E}\cdot\overrightarrow{D}
\end{equation*}
\item Raumladungsdichte
\begin{equation*}
\rho(\overrightarrow{r})=q\cdot n(\overrightarrow{r})
\end{equation*}
$n$: Anzahl Ladungsträger pro Volumen
\item Unendlich langer zylindrischer Leiter:
\begin{equation*}
\overrightarrow{E}(\overrightarrow{r})=\frac{q}{2\pi\epsilon}\cdot\frac{\overrightarrow{r}-\overrightarrow{r}_0}{|\overrightarrow{r}-\overrightarrow{r}_0|^2}
\end{equation*}
\begin{equation*}
\Phi(\overrightarrow{r})=-\frac{q}{2\pi\epsilon}\cdot ln|\overrightarrow{r}-\overrightarrow{r}_0|
\end{equation*}
\end{enumerate}

\subsection{Stationäre Ströme}
\begin{enumerate}[label=$\bullet$]
\item
\begin{equation*}
I_A=\left.\frac{dQ}{dt}\right|_A
\end{equation*}
\item
\begin{equation*}
I(A)=\int\limits_{A}\overrightarrow{j}d\overrightarrow{a}
\end{equation*}
$\overrightarrow{j}$: Elektrische Stromdichte
\item
\begin{equation*}
\overrightarrow{j}=\sum\limits_{\alpha=1}^{K}q_\alpha\cdot n_\alpha\cdot\overrightarrow{v}_\alpha=\sum\limits_{\alpha=1}^{K}|q_\alpha |\cdot n_\alpha\cdot\mu_\alpha\cdot\overrightarrow{E}
\end{equation*}
$\mu_\alpha$: Beweglichkeit
\item Mittlere Driftgeschwindigkeit
\begin{equation*}
\overrightarrow{v}=\frac{q\cdot\tau}{m^*}\cdot\overrightarrow{E}=sgn(q)\mu\overrightarrow{E}
\end{equation*}
$\tau$: mittlere Stoßzeit\\
$m^*$: effektive Masse\\
$\mu$: Beweglichkeit
\begin{equation*}
\mu=\frac{|q|\tau}{m^*}
\end{equation*}
\item Spezifische el. Leitfähigkeit
\begin{equation*}
\sigma=\sum\limits_{\alpha=1}^{k}|q_\alpha |n_\alpha\mu_\alpha
\end{equation*}
\item Lokales Ohmsches Gesetz:
\begin{equation*}
\overrightarrow{j}=\sigma\cdot\overrightarrow{E}
\end{equation*}
\item Spezifischer el. Widerstand:
\begin{equation*}
\rho =\frac{1}{\sigma}
\end{equation*}
\item Elektrischer Leitwert
\begin{equation*}
G=\sigma\cdot\frac{A}{l}
\end{equation*}
\item Elektrischer Widerstand:
\begin{equation*}
R=\rho\cdot\frac{l}{A}
\end{equation*}
\item Ohmsches Gesetz
\begin{equation*}
U=R\cdot I
\end{equation*}
\item KCL:
\begin{equation*}
0=\int\limits_{\partial V}\overrightarrow{j}d\overrightarrow{a}=\sum\limits_{k=1}^{N}\int\limits_{A_k}\overrightarrow{j}d\overrightarrow{a}=\sum\limits_{k=1}^{N}I_k
\end{equation*}
\item Ladungsbilanzgleichung (integrale Form):
\begin{equation*}
\frac{dQ(V)}{dt}=-\int\limits_{\partial V}\overrightarrow{j}d\overrightarrow{a}
\end{equation*}
\item Ladungsbilanzgleichung (differentielle Form):
\begin{equation*}
\nabla\overrightarrow{j}+\frac{\partial \rho}{\partial t}=0
\end{equation*}
\item El. Leistung einer Punktladung:
\begin{equation*}
P_{el}=q\cdot\overrightarrow{E}\cdot\overrightarrow{v}
\end{equation*}
\item El. Leistungsdichte:
\begin{equation*}
p_{el}=\overrightarrow{j}\cdot\overrightarrow{E}
\end{equation*}
\item El. Leistung:
\begin{equation*}
P_{el}=\int\limits_{V}p_{el}(\overrightarrow{r})dv=U\cdot I=\frac{U^2}{R}=I^2\cdot R
\end{equation*}
\item El. Energieübertragungsstrecke:
\begin{equation*}
\eta =1-\frac{R_L\cdot P_E}{U_E^2}
\end{equation*}
\end{enumerate}

\subsection{Magnetostatik}
\begin{enumerate}[label=$\bullet$]
\item Lorentzkraftdichte:
\begin{equation*}
\overrightarrow{f}_L=\overrightarrow{j}\times\overrightarrow{B}
\end{equation*}
\begin{equation*}
\overrightarrow{f}_{em}=\rho\cdot\overrightarrow{E}+\overrightarrow{j}\times\overrightarrow{B}
\end{equation*}
$\rho$: Raumladungsdichte
\item Lorentzkraft:
\begin{equation*}
\overrightarrow{F}_{Leiter}=\int\limits_{Leiter}\overrightarrow{f}_L d^3r
\end{equation*}
\begin{equation*}
\overrightarrow{F}_L=q(\overrightarrow{v}\times\overrightarrow{B})
\end{equation*}
\item Kraft auf linienförmige Leiter:
\begin{equation*}
d\overrightarrow{F}_L=I\cdot d\overrightarrow{s}\times\overrightarrow{B}
\end{equation*}
\begin{equation*}
\overrightarrow{F}_{Leiter}=\int\limits_{C}d\overrightarrow{F}_L
\end{equation*}
\item Elektromagn. Kraft:
\begin{equation*}
\overrightarrow{F}_{em}=q(\overrightarrow{E}+\overrightarrow{v}\times\overrightarrow{B})
\end{equation*}
\item Amperesches Durchflutungsgesetz:
\begin{equation*}
I(A)=\int\limits_{\partial A}\overrightarrow{H}d\overrightarrow{r}
\end{equation*}
\item Magnetfeld von unendlich langem Leiter:
\begin{equation*}
\overrightarrow{H}(\overrightarrow{r})=\frac{I}{2\pi r}\overrightarrow{e}_\varphi
\end{equation*}
\item Kraft auf Leiter beliebiger Gestalt:
\begin{equation*}
\overrightarrow{F}_{Leiter}=\int\limits_{Leiter}\overrightarrow{j}(\overrightarrow{r})\times\overrightarrow{B}(\overrightarrow{r})d^3r
\end{equation*}
\item Kraft auf linienförmige Leiter:
\begin{equation*}
\overrightarrow{F}_{Leiter}=\int\limits_{C}d\overrightarrow{F}_L;\;\;d\overrightarrow{F}_L=I\cdot d\overrightarrow{s}\times\overrightarrow{B}
\end{equation*}
\item Magnetisches Moment
\begin{equation*}
\overrightarrow{m}=I\cdot\overrightarrow{A}
\end{equation*}
\begin{equation*}
\overrightarrow{m}=V\cdot\mathcal{M}
\end{equation*}
$\mathcal{M}$: Magnetisierung
\item Drehmoment:
\begin{equation*}
\overrightarrow{M}=(\overrightarrow{r}-\overrightarrow{r}_0)\times\overrightarrow{F}=\overrightarrow{m}\times\overrightarrow{B}
\end{equation*}
\item Permanentmagnet:
\begin{equation*}
\mathcal{M}=n\cdot <\overrightarrow{m_0}>
\end{equation*}
$<\overrightarrow{m_0}>$: Mittelwert der atomaren magn. Momente\\
$n$: Anzahl pro Volumen\\
\item 
\begin{equation*}
\mathcal{M}=\mathcal{X}_m\cdot\overrightarrow{H}
\end{equation*}
\begin{equation*}
\mu_r=1+\mathcal{X}_m
\end{equation*}
$\mathcal{X}_m$: magnetische Suszeptibilität
\end{enumerate}

\subsection{Induktion}
\begin{enumerate}[label=$\bullet$]
\item Magnetischer Fluss:
\begin{equation*}
\Phi_{mag}=\int\limits_{A}\overrightarrow{B}d\overrightarrow{a}
\end{equation*}
\item Induktionsspannung:
\begin{equation*}
U_{ind}=-\dot{\Phi}_{mag}
\end{equation*}
\begin{equation*}
\overrightarrow{E}_{ind}=\overrightarrow{v}\times\overrightarrow{B}
\end{equation*}
\begin{equation*}
U_{ind}=\underbrace{-\int\limits_{A(t)}\frac{\partial \overrightarrow{B}}{\partial t}d\overrightarrow{a}}_{Ruheinduktion}+\underbrace{\int\limits_{\partial A(t)}\overrightarrow{v}\times\overrightarrow{B}d\overrightarrow{r}}_{Bewegungsinduktion}=-\dot{\Phi(A(t))}
\end{equation*}
\end{enumerate}

\subsection{Maxwellsche Gleichungen}

\subsubsection{Integrale Form}
\begin{enumerate}[label=$\bullet$]
\item
\begin{equation*}
\int\limits_{\partial V}\overrightarrow{D}d\overrightarrow{a}=\int\limits_{V}\nabla\overrightarrow{D}dv=Q(V)=\int\limits_{V}\rho(\overrightarrow{r})dv
\end{equation*}
\item
\begin{equation*}
\int\limits_{\partial V}\overrightarrow{B}d\overrightarrow{a}=0
\end{equation*}
\item
\begin{equation*}
\int\limits_{\partial A}\overrightarrow{E}_{ind,r}d\overrightarrow{r}=-\int\limits_{A}\frac{\partial\overrightarrow{B}}{\partial t}d\overrightarrow{a}
\end{equation*}
\item Erweiterung des Ampere'schen Gesetzes:
\begin{equation*}
\int\limits_{\partial A}\overrightarrow{H}d\overrightarrow{r}=\int\limits_{A}\overrightarrow{j}+\frac{\partial \overrightarrow{D}}{\partial t}d\overrightarrow{a}
\end{equation*}
\end{enumerate}

\subsubsection{Differentielle Form}
\begin{enumerate}[label=$\bullet$]
\item Gauß'sches Gesetz
\begin{equation*}
\nabla\overrightarrow{D}=\rho
\end{equation*}
\item Quellenfreiheit des $\overrightarrow{B}$-Feldes
\begin{equation*}
\nabla\overrightarrow{B}=0
\end{equation*}
\item Faraday'sches Induktionsgesetz
\begin{equation*}
rot(\overrightarrow{E})=-\frac{\partial \overrightarrow{B}}{\partial t}
\end{equation*}
\item Ampere'sches Durchflutungsgesetz
\begin{equation*}
rot(\overrightarrow{H})=\overrightarrow{j}+\frac{\partial \overrightarrow{D}}{\partial t}
\end{equation*}
\end{enumerate}

\subsection{Materialgleichungen}
\begin{enumerate}[label=$\bullet$]
\item
\begin{equation*}
\overrightarrow{D}=\epsilon\overrightarrow{E}
\end{equation*}
\item
\begin{equation*}
\overrightarrow{B}=\mu\overrightarrow{H}
\end{equation*}
\item
\begin{equation*}
\overrightarrow{j}=\sigma\overrightarrow{E}
\end{equation*}
\end{enumerate}

\subsection{Einheiten}
\begin{enumerate}[label=$\bullet$]
\item Elektrische Leitfähigkeit
\begin{equation*}
[\sigma]=\frac{1}{\Omega m}=\frac{A^2s^3}{m^3kg}
\end{equation*}
\item Beweglichkeit
\begin{equation*}
[\mu]=\frac{m^2}{Vs}=\frac{As^2}{kg}
\end{equation*}
\item Absolute Permittivität
\begin{equation*}
[\epsilon_0]=\frac{As}{Vm}
\end{equation*}
\item Spannung
\begin{equation*}
[V]=\frac{kg\cdot m^2}{As^3}
\end{equation*}
\item Absolute Permeabilität
\begin{equation*}
[\mu_0]=\frac{\Omega s}{m}=\frac{Vs}{Am}
\end{equation*}
\item Magnetische Flussdichte
\begin{equation*}
[B]=T=\frac{Vs}{m^2}
\end{equation*}
\end{enumerate}

\subsection{Sonstiges}
\begin{enumerate}[label=$\bullet$]
\item Gyrationsfrequenz:
\begin{equation*}
\Omega =\frac{q\cdot B}{m}
\end{equation*}
\item Stetigkeit
\begin{enumerate}[label=]
\item $\overrightarrow{E}$: Tangentialkomponente stetig
\item $\overrightarrow{D}$: Normalkomponente stetig
\item $\overrightarrow{B}$: Normalkomponente stetig
\item $\overrightarrow{H}$: Tangentialkomponente stetig
\end{enumerate}
\end{enumerate}
Lizenz: CC BY-NC-SA 3.0\\
\url{http://creativecommons.org/licenses/by-nc-sa/3.0/de/}

\end{document}








